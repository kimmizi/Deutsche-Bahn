%%%%%%%% ICML 2023 EXAMPLE LATEX SUBMISSION FILE %%%%%%%%%%%%%%%%%

\documentclass{article}

% Recommended, but optional, packages for figures and better typesetting:
\usepackage{microtype}
\usepackage{graphicx}
\usepackage{subfigure}
\usepackage{booktabs} % for professional tables

\usepackage{tikz}
% Corporate Design of the University of Tübingen
% Primary Colors
\definecolor{TUred}{RGB}{165,30,55}
\definecolor{TUgold}{RGB}{180,160,105}
\definecolor{TUdark}{RGB}{50,65,75}
\definecolor{TUgray}{RGB}{175,179,183}

% Secondary Colors
\definecolor{TUdarkblue}{RGB}{65,90,140}
\definecolor{TUblue}{RGB}{0,105,170}
\definecolor{TUlightblue}{RGB}{80,170,200}
\definecolor{TUlightgreen}{RGB}{130,185,160}
\definecolor{TUgreen}{RGB}{125,165,75}
\definecolor{TUdarkgreen}{RGB}{50,110,30}
\definecolor{TUocre}{RGB}{200,80,60}
\definecolor{TUviolet}{RGB}{175,110,150}
\definecolor{TUmauve}{RGB}{180,160,150}
\definecolor{TUbeige}{RGB}{215,180,105}
\definecolor{TUorange}{RGB}{210,150,0}
\definecolor{TUbrown}{RGB}{145,105,70}

% hyperref makes hyperlinks in the resulting PDF.
% If your build breaks (sometimes temporarily if a hyperlink spans a page)
% please comment out the following usepackage line and replace
% \usepackage{icml2023} with \usepackage[nohyperref]{icml2023} above.
\usepackage{hyperref}


% Attempt to make hyperref and algorithmic work together better:
\newcommand{\theHalgorithm}{\arabic{algorithm}}

\usepackage[accepted]{icml2023}

% For theorems and such
\usepackage{amsmath}
\usepackage{amssymb}
\usepackage{mathtools}
\usepackage{amsthm}

% if you use cleveref..
\usepackage[capitalize,noabbrev]{cleveref}

%%%%%%%%%%%%%%%%%%%%%%%%%%%%%%%%
% THEOREMS
%%%%%%%%%%%%%%%%%%%%%%%%%%%%%%%%
\theoremstyle{plain}
\newtheorem{theorem}{Theorem}[section]
\newtheorem{proposition}[theorem]{Proposition}
\newtheorem{lemma}[theorem]{Lemma}
\newtheorem{corollary}[theorem]{Corollary}
\theoremstyle{definition}
\newtheorem{definition}[theorem]{Definition}
\newtheorem{assumption}[theorem]{Assumption}
\theoremstyle{remark}
\newtheorem{remark}[theorem]{Remark}

% Todonotes is useful during development; simply uncomment the next line
%    and comment out the line below the next line to turn off comments
%\usepackage[disable,textsize=tiny]{todonotes}
\usepackage[textsize=tiny]{todonotes}


% The \icmltitle you define below is probably too long as a header.
% Therefore, a short form for the running title is supplied here:
\icmltitlerunning{Abstract Template for Data Literacy 2023/24}

\begin{document}

\twocolumn[
\icmltitle{Abstract Template for Data Literacy 2023/24\\ Deutsche Bahn - Predicting and Visualising the One Exception to the German Punctuality Rule}

% It is OKAY to include author information, even for blind
% submissions: the style file will automatically remove it for you
% unless you've provided the [accepted] option to the icml2023
% package.

% List of affiliations: The first argument should be a (short)
% identifier you will use later to specify author affiliations
% Academic affiliations should list Department, University, City, Region, Country
% Industry affiliations should list Company, City, Region, Country

% You can specify symbols, otherwise they are numbered in order.
% Ideally, you should not use this facility. Affiliations will be numbered
% in order of appearance and this is the preferred way.
\icmlsetsymbol{equal}{*}

\begin{icmlauthorlist}
\icmlauthor{Jiha Kim}{equal,first}
\icmlauthor{Kim Zierahn}{equal,second}
\icmlauthor{Sumitrra Bala Subramaniam}{equal,third}
\icmlauthor{Frederik Panse}{equal,fourth}
\end{icmlauthorlist}

% fill in your matrikelnummer, email address, degree, for each group member
\icmlaffiliation{first}{Matrikelnummer 6640082, jiha.kim@student.uni-tuebingen.de, MSc Quantitative Data Science Methods}
\icmlaffiliation{second}{Matrikelnummer 6635183, kim-isabella.zierahn@student.uni-tuebingen.de, MSc Quantitative Data Science Methods}
\icmlaffiliation{third}{Matrikelnummer 6642271, sumitrra.bala-subramaniam@student.uni-tuebingen.de, MSc Quantitative Data Science Methods}
\icmlaffiliation{fourth}{Matrikelnummer 5810899, frederik.panse@student.uni-tuebingen.de, MSc Quantitative Data Science Methods}

% You may provide any keywords that you
% find helpful for describing your paper; these are used to populate
% the "keywords" metadata in the PDF but will not be shown in the document
\icmlkeywords{Machine Learning}

\vskip 0.3in
]   

% this must go after the closing bracket ] following \twocolumn[ ...

% This command actually creates the footnote in the first column
% listing the affiliations and the copyright notice.
% The command takes one argument, which is text to display at the start of the footnote.
% The \icmlEqualContribution command is standard text for equal contribution.
% Remove it (just {}) if you do not need this facility.

%\printAffiliationsAndNotice{}  % leave blank if no need to mention equal contribution
\printAffiliationsAndNotice{\icmlEqualContribution} % otherwise use the standard text.

\begin{abstract}
The reputation of Deutsche Bahn (DB) raises questions regarding its reliability in maintaining schedules. This project aims to predict the punctuality of trains operated by DB with respect to both time and geographical locations. We are planning to use data provided by the \href{https://data.deutschebahn.com/dataset/ist-verkehrsdaten-der-db-cargo-auf-bst8-ebene.html}{Deutsche Bahn} in 2016 and the corresponding \href{https://data.deutschebahn.com/dataset/betriebsstellen-gueterverkehr.html}{directory of operating sites}. We came across this dataset on recorded train schedules and delays in 2016 on Deutsche Bahn's website as some of our friends and other members of the community in Tübingen are affected by the train delays almost on a day-to-day basis. We are planning to do a linear regression, predicting the minutes of delay. Our features will include the coordinates, the month, the (week)day and the number of trains. We are planning to do a visualization in form of a map, depicting the delay in Germany. We are expecting to identify and predict operating locations or regions with high or low delay minutes. 

\end{abstract}


\section{Contribution Plan}

We are planning to work as a team on this project in all phases. However, Kim Zierahn is going to do the data cleaning and EDA. Jiha Kim and Sumitrra Bala Subramaniam are going to do the predictive model. Frederik Panse is going to do the data visualization. All authors will jointly write the text of the report. 


\bibliography{bibliography}
\bibliographystyle{icml2023}

\end{document}


% This document was modified from the file originally made available by
% Pat Langley and Andrea Danyluk for ICML-2K. This version was created
% by Iain Murray in 2018, and modified by Alexandre Bouchard in
% 2019 and 2021 and by Csaba Szepesvari, Gang Niu and Sivan Sabato in 2022.
% Modified again in 2023 by Sivan Sabato and Jonathan Scarlett.
% Previous contributors include Dan Roy, Lise Getoor and Tobias
% Scheffer, which was slightly modified from the 2010 version by
% Thorsten Joachims & Johannes Fuernkranz, slightly modified from the
% 2009 version by Kiri Wagstaff and Sam Roweis's 2008 version, which is
% slightly modified from Prasad Tadepalli's 2007 version which is a
% lightly changed version of the previous year's version by Andrew
% Moore, which was in turn edited from those of Kristian Kersting and
% Codrina Lauth. Alex Smola contributed to the algorithmic style files.
